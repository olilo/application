%%%%%%%%%%%%%%%%%%%%%%%%%%%%%%%%%%%%%%%%%
% "ModernCV" CV and Cover Letter
% LaTeX Template
% Version 1.1 (9/12/12)
%
% This template has been downloaded from:
% http://www.LaTeXTemplates.com
%
% Original author:
% Xavier Danaux (xdanaux@gmail.com)
%
% License:
% CC BY-NC-SA 3.0 (http://creativecommons.org/licenses/by-nc-sa/3.0/)
%
% Important note:
% This template requires the moderncv.cls and .sty files to be in the same 
% directory as this .tex file. These files provide the resume style and themes 
% used for structuring the document.
%
%%%%%%%%%%%%%%%%%%%%%%%%%%%%%%%%%%%%%%%%%

%----------------------------------------------------------------------------------------
%	PACKAGES AND OTHER DOCUMENT CONFIGURATIONS
%----------------------------------------------------------------------------------------

\documentclass[11pt,a4paper,sans]{moderncv} % Font sizes: 10, 11, or 12; paper sizes: a4paper, letterpaper, a5paper, legalpaper, executivepaper or landscape; font families: sans or roman

\usepackage[utf8]{inputenc}

\usepackage[ngerman]{babel}

\moderncvstyle{classic} % CV theme - options include: 'casual' (default), 'classic', 'oldstyle' and 'banking'
\moderncvcolor{blue} % CV color - options include: 'blue' (default), 'orange', 'green', 'red', 'purple', 'grey' and 'black'

\usepackage{lipsum} % Used for inserting dummy 'Lorem ipsum' text into the template

\usepackage[scale=0.75]{geometry} % Reduce document margins
\setlength{\hintscolumnwidth}{4cm} % Uncomment to change the width of the dates column
%\setlength{\makecvtitlenamewidth}{10cm} % For the 'classic' style, uncomment to adjust the width of the space allocated to your name

%----------------------------------------------------------------------------------------
%   NAME AND CONTACT INFORMATION SECTION
%----------------------------------------------------------------------------------------

\firstname{Oliver} % Your first name
\familyname{Lorenz} % Your last name

% All information in this block is optional, comment out any lines you don't need
\title{Lebenslauf}
\address{Auerstraße 27}{10249 Berlin}
\mobile{+49 176 57264258}
\email{oliver-lorenz@gmx.de}

\homepage{www.github.com/olilo}{} % The first argument is the url for the clickable link, the second argument is the url displayed in the template - this allows special characters to be displayed such as the tilde in this example

%\extrainfo{additional information}

\photo[75pt][0pt]{pictures/profile} % The first bracket is the picture height, the second is the thickness of the frame around the picture (0pt for no frame)
%\quote{"Mein Fulli pusselt." - Willi Schönborn}


\begin{document}

\makecvtitle % Print the CV title

%----------------------------------------------------------------------------------------
%	PERSONAL INFORMATION SECTION
%----------------------------------------------------------------------------------------

\section{Persönliche Angaben}

\cvitem{Geburtsdatum}{31. Januar 1987}
\cvitem{Staatsangehörigkeit}{deutsch}
\cvitem{Familienstand}{ledig, keine Kinder}

%----------------------------------------------------------------------------------------
%	WORK EXPERIENCE SECTION
%----------------------------------------------------------------------------------------

\section{Berufserfahrung}

\cventry{09/2007 -- Heute}{Senior Software Engineer}{\textsc{CosmoCode GmbH}}{Berlin}{\begin{itemize}
\item Entwicklung von Java- und PHP-Web-Applikationen
\item Agile Projektmethoden (Scrum und Kanban)
\item Technisches Projektmanagement
\item Technische Konzeption
\item Kundenkommunikation
\item Performance-Optimierung
\item Support
\item Qualitätssicherung
\end{itemize}}

%----------------------------------------------------------------------------------------
%	EDUCATION SECTION
%----------------------------------------------------------------------------------------

\section{Ausbildung}

\cventry{08/2012 -- 07/2013}{Fachinformatiker für Anwendungsentwicklung}{IHK}{Berlin}{\textit{Durchschnittsnote -- 3??}}
\cventry{06/2006 -- 09/2010}{}{Informatik}{Freie Universität}{Berlin}{}
\cventry{03/2001 -- 06/2006}{Allgemeine Hochschulreife}{Abitur}{Heinrich-Hertz-Oberschule}{Berlin}{\textit{Durchschnittsnote -- 2.7}}
\cventry{09/1997 -- 03/2001}{}{}{Johann-Gottfried-Herder Gymnasium}{Berlin}{}
\cventry{08/1993 -- 07/1997}{}{Grundschule}{Grundschule am Beerenpfuhl}{Berlin}{}

%----------------------------------------------------------------------------------------
%	PROJECTS SECTION
%----------------------------------------------------------------------------------------
\section{Referenzen}

\subsection{Palava}
\cvitem{Art}{Open Source}
\cvitem{Jahr}{2010 -- 2012}
\cvitem{Team}{Tobias Sarnowski \& Willi Schönborn}
\cvitem{Quellcode}{\url{https://github.com/palava}}
\cvitem{Technologien}{\textsc{Java}, \textsc{Guice}, \textsc{JBoss Netty}}
\cvitem{Beschreibung}{Palava ist ein extrem modulares Application-Framework für Java-Backends. Der zentrale Bestandteil ist eine mächtige Interprocess-Communication-API, die es erlaubt einheitliche Endpoints zu entwickeln die dann mit unterschiedlichsten Transport-Technologien (z.B. \textsc{SOAP}, \textsc{REST}, \textsc{RMI}, \textsc{XML-RPC}, \textsc{JSON-RPC}, u.a.) genutzt werden können.}

\vspace{10pt}

\subsection{TheLabelFinder}
\cvitem{Art}{Kunden-Projekt}
\cvitem{Jahr}{2008 -- 2010}
\cvitem{URL}{\url{http://www.thelabelfinder.com}}
\cvitem{Technologien}{\textsc{Palava}, \textsc{Flash AS2}, \textsc{Java}, \textsc{PHP}, \textsc{Solr}, \textsc{MySQL}, \textsc{Apache}, \textsc{Nginx}}
\cvitem{Beschreibung}{TheLabelFinder ist eine Fashion-Search-Engine auf Basis des Palava-Frameworks.}

\vspace{10pt}

\subsection{Spider Briber}
\cvitem{Art}{Freizeit-Projekt}
\cvitem{Jahr}{2013}
\cvitem{Technologien}{\textsc{Javascript}, \textsc{Canvas}, \textsc{HTML5}}
\cvitem{Quellcode}{\url{https://github.com/olilo/spider-briber}}
\cvitem{Beschreibung}{Ein HTML5 Spiele-Prototyp, den ich während der \url{http://www.onegameamonth.com} Challenge erstellt habe.}

\vspace{10pt}

\subsection{Steganography}
\cvitem{Art}{Semester-Projekt}
\cvitem{Jahr}{2001}
\cvitem{Technologien}{\textsc{Python}, \textsc{HMAC}, \textsc{XTEA}, \textsc{CFB}}
\cvitem{Quellcode}{\url{https://github.com/whiskeysierra/steganography}}
\cvitem{Beschreibung}{Ein Python-Programm zum Verschlüsseln und Einbetten von Text-Nachrichten in Bitmaps.}

\vspace{10pt}

\subsection{Impaired Vision}
\cvitem{Art}{Semester-Projekt}
\cvitem{Jahr}{2011}
\cvitem{Technologien}{\textsc{Java}, \textsc{Android}}
\cvitem{Quellcode}{\url{https://github.com/whiskeysierra/impaired-vision}}
\cvitem{Beschreibung}{Eine Augmented Reality App zur Simulation von verschiedenen Farb-Sehstörungen für die Android-Plattform.}

\newpage
%----------------------------------------------------------------------------------------
%	TECHNOLOGIES SECTION
%----------------------------------------------------------------------------------------

\section{Technologien}
Mit den folgenden Technologien arbeite ich täglich bis regelmäßig:
\newline

\cvitem{Sprachen}{\textsc{Java}, \textsc{SQL}, \textsc{JavaScript}, \textsc{AspectJ}, \textsc{PHP}, \textsc{HTML}, \textsc{Python}, \textsc{CSS}, \LaTeX, \textsc{Shell}}
\cvitem{Data Stores}{\textsc{Solr}, \textsc{MySQL}, \textsc{PostgreSQL}}
\cvitem{MapReduce}{\textsc{Hadoop}, \textsc{Hive}}
\cvitem{Libraries}{\textsc{Guava}, \textsc{SLF4J}, \textsc{jQuery}, \textsc{Underscore}}
\cvitem{Frameworks}{\textsc{Guice}, \textsc{JPA}, \textsc{Servlet API}, \textsc{Backbone}, \textsc{Symfony2}}
\cvitem{Testing}{\textsc{Junit}, \textsc{EasyMock}, \textsc{Jukito}}
\cvitem{SCM}{\textsc{Git}, \textsc{Subversion}}
\cvitem{Build Tools}{\textsc{Ant}, \textsc{Maven}}

%----------------------------------------------------------------------------------------
%	LANGUAGES SECTION
%----------------------------------------------------------------------------------------

\section{Sprachkenntnisse}

\cvitemwithcomment{Deutsch}{Muttersprache}{}
\cvitemwithcomment{English}{Fließend}{}

%----------------------------------------------------------------------------------------

\end{document}
