%%%%%%%%%%%%%%%%%%%%%%%%%%%%%%%%%%%%%%%%%
% "ModernCV" CV and Cover Letter
% LaTeX Template
% Version 1.1 (9/12/12)
%
% This template has been downloaded from:
% http://www.LaTeXTemplates.com
%
% Original author:
% Xavier Danaux (xdanaux@gmail.com)
%
% License:
% CC BY-NC-SA 3.0 (http://creativecommons.org/licenses/by-nc-sa/3.0/)
%
% Important note:
% This template requires the moderncv.cls and .sty files to be in the same 
% directory as this .tex file. These files provide the resume style and themes 
% used for structuring the document.
%
%%%%%%%%%%%%%%%%%%%%%%%%%%%%%%%%%%%%%%%%%

%----------------------------------------------------------------------------------------
%	PACKAGES AND OTHER DOCUMENT CONFIGURATIONS
%----------------------------------------------------------------------------------------

\documentclass[11pt,a4paper,sans]{moderncv} % Font sizes: 10, 11, or 12; paper sizes: a4paper, letterpaper, a5paper, legalpaper, executivepaper or landscape; font families: sans or roman

\usepackage[utf8]{inputenc}

\usepackage[ngerman]{babel}

\moderncvstyle{classic} % CV theme - options include: 'casual' (default), 'classic', 'oldstyle' and 'banking'
\moderncvcolor{blue} % CV color - options include: 'blue' (default), 'orange', 'green', 'red', 'purple', 'grey' and 'black'

\usepackage{lipsum} % Used for inserting dummy 'Lorem ipsum' text into the template

\usepackage[scale=0.75]{geometry} % Reduce document margins
\setlength{\hintscolumnwidth}{4cm} % Uncomment to change the width of the dates column
%\setlength{\makecvtitlenamewidth}{10cm} % For the 'classic' style, uncomment to adjust the width of the space allocated to your name

%----------------------------------------------------------------------------------------
%   NAME AND CONTACT INFORMATION SECTION
%----------------------------------------------------------------------------------------

\firstname{Oliver} % Your first name
\familyname{Lorenz} % Your last name

% All information in this block is optional, comment out any lines you don't need
\title{Lebenslauf}
\address{Rigaer Straße 58}{10247 Berlin}
\mobile{+49 176 57264258}
\phone{+49 30 ????}
\email{endurielfm@gmail.com}

\homepage{www.github.com/olilo}{} % The first argument is the url for the clickable link, the second argument is the url displayed in the template - this allows special characters to be displayed such as the tilde in this example

%\extrainfo{additional information}

\photo[75pt][0pt]{pictures/profile} % The first bracket is the picture height, the second is the thickness of the frame around the picture (0pt for no frame)
%\quote{"Mein Fulli pusselt." - Willi Schönborn}


\begin{document}

\makecvtitle % Print the CV title

%----------------------------------------------------------------------------------------
%	PERSONAL INFORMATION SECTION
%----------------------------------------------------------------------------------------

\section{Persönliche Angaben}

\cvitem{Geburtsdatum}{31. Januar 1987}
\cvitem{Staatsangehörigkeit}{deutsch}
\cvitem{Familienstand}{ledig, keine Kinder}

%----------------------------------------------------------------------------------------
%	WORK EXPERIENCE SECTION
%----------------------------------------------------------------------------------------

\section{Berufserfahrung}

\cventry{09/2010 -- Heute}{Senior Software Engineer}{}{\textsc{CosmoCode GmbH}}{Berlin}{\begin{itemize}
\item Entwicklung von Java-, PHP- und Python-Web-Applikationen
\item Agile Projektmethoden (Scrum und Kanban)
\item Technisches Projektmanagement
\item Technische Konzeption
\item Kundenkommunikation
\item Performance-Optimierung
\item Support
\item Qualitätssicherung
\end{itemize}}


\cventry{09/2007 -- 09/2010}{Software Engineer Trainee}{}{\textsc{CosmoCode GmbH}}{Berlin}{\begin{itemize}
\item Entwicklung von Java- und PHP-Web-Applikationen
\item Agile Projektmethoden (Scrum)
\item Technische Konzeption
\item Performance-Optimierung
\item Support
\item Qualitätssicherung
\end{itemize}}

%----------------------------------------------------------------------------------------
%	EDUCATION SECTION
%----------------------------------------------------------------------------------------

\section{Ausbildung}

\cventry{08/2012 -- 07/2013}{Fachinformatiker für Anwendungsentwicklung}{Externe Ausbildung}{IHK}{Berlin}{\textit{Durchschnittsnote -- 75 Punkte}}
\cventry{10/2006 -- 09/2010}{\textit{kein Abschluss}}{Informatik}{Freie Universität}{Berlin}{}
\cventry{03/2001 -- 07/2006}{Allgemeine Hochschulreife}{Abitur}{Heinrich-Hertz-Oberschule}{Berlin}{\textit{Durchschnittsnote -- 2.7}, Leistungskurse: \textit{Mathematik, Informatik}}
\cventry{09/1997 -- 03/2001}{-}{-}{Johann-Gottfried-Herder Gymnasium}{Berlin}{}

\newpage

%----------------------------------------------------------------------------------------
%	PROJECTS SECTION
%----------------------------------------------------------------------------------------
\section{Referenzen}

\vspace{10pt}

\subsection{Flower Tower Defense}
\cvitem{Art}{Freizeit-Projekt, Spiel}
\cvitem{Jahr}{2013}
\cvitem{URL}{\url{http://olilo.github.io/flower-tower-defense}}
\cvitem{Technologien}{\textsc{Javascript}, \textsc{Canvas}, \textsc{HTML5}, \textsc{Crafty}}
\cvitem{Quellcode}{\url{https://github.com/olilo/flower-tower-defense}}
\cvitem{Beschreibung}{Ein Prototyp eines in meiner Freizeit erstellten HTML5 Tower Defense Spiel, bei dem  Blumentürme auf Hexen schießen.}

\vspace{10pt}

\subsection{Spider Briber}
\cvitem{Art}{Freizeit-Projekt, Spiel}
\cvitem{Jahr}{2013}
\cvitem{URL}{\url{http://olilo.github.io/spider-briber/}}
\cvitem{Technologien}{\textsc{Javascript}, \textsc{Canvas}, \textsc{HTML5}}
\cvitem{Quellcode}{\url{https://github.com/olilo/spider-briber}}
\cvitem{Beschreibung}{Ein HTML5 Spiele-Prototyp, den ich während der \url{http://www.onegameamonth.com} Challenge erstellt habe.}

\vspace{10pt}

\subsection{Hightech-Strategie: Gesundheit \& Ernährung}
\cvitem{Art}{Kundenprojekt}
\cvitem{Jahr}{2013}
\cvitem{URL}{\url{http://www.zukunft-forschung.de/gesundheit/}}
\cvitem{Team}{Andreas Gohr}
\cvitem{Technologien}{\textsc{Javascript}, \textsc{jQuery}, \textsc{jQuery Stellar}, \textsc{HTML5}}
\cvitem{Beschreibung}{Eine Webseite, die eine gesunde Lebensweise bis ins hohe Alter auf eine spielerische Art vermitteln will. Umgesetzt wurde dies mit Parallax-Scrolling, welches mithilfe des jQuery Plugins \textsc{Stellar} erreicht wurde.}

\vspace{10pt}

\subsection{YourChance}
\cvitem{Art}{Kundenprojekt}
\cvitem{Jahr}{2012}
\cvitem{URL}{\url{http://www.yourchance.de/}}
\cvitem{Team}{Tobias Sarnowski, Florian Purchess \& Jonas Jurczok}
\cvitem{Technologien}{\textsc{Java}, \textsc{JBoss 7}, \textsc{JSF2}, \textsc{Youtube API}, \textsc{Amazon S3}}
\cvitem{Beschreibung}{Eine online Casting-Plattform, bei der unter anderem Models und Schauspieler sich auf diverse Castings, z.B. für deutsche Fernsehserien oder Shows à la DSDS bewerben können. Videos werden auf Youtube hochgeladen und Bilder auf Amazon S3 gelagert.}

\newpage

\subsection{Palava}
\cvitem{Art}{Open Source}
\cvitem{Jahr}{2010 -- 2012}
\cvitem{Team}{Tobias Sarnowski \& Willi Schönborn}
\cvitem{Quellcode}{\url{https://github.com/palava}}
\cvitem{Technologien}{\textsc{Java}, \textsc{Guice}, \textsc{JBoss Netty}}
\cvitem{Beschreibung}{Palava ist ein extrem modulares Application-Framework für Java-Backends. Der zentrale Bestandteil ist eine mächtige Interprocess-Communication-API, die es erlaubt einheitliche Endpoints zu entwickeln die dann mit unterschiedlichsten Transport-Technologien (z.B. \textsc{SOAP}, \textsc{REST}, \textsc{RMI}, \textsc{XML-RPC}, \textsc{JSON-RPC}, u.a.) genutzt werden können.}

\vspace{10pt}

\subsection{SPD.de}
\cvitem{Art}{Kundenprojekt, Webseite}
\cvitem{Jahr}{2009}
\cvitem{URL}{mittlerweile offline wegen politisch motiviertem Relaunch}
\cvitem{Team}{Tobias Sarnowski \& Florian Purchess \& Marco Zanter}
\cvitem{Technologien}{\textsc{Java}, \textsc{Lucene}, \textsc{JSP}, \textsc{MySQL}}
\cvitem{Beschreibung}{Die Hauptseite der SPD wurde mit einem von der Firma CosmoCode GmbH erstellten CMS umgesetzt. Dieses CMS wurde in \textsc{Java} umgesetzt. Dazu gab es noch eine Suche mit Wortvervollständigung, die in \textsc{Lucene} umgesetzt wurde. Um dem Besucheransturm auf die Seite standhalten zu können wurden Live-Auftritt und die Suche auf 4 Rechnern geclustered. Meine Hauptaufgabe war dabei alles, was die Suche betraf, inklusive Clustering.}

\vspace{10pt}

\subsection{TheLabelFinder}
\cvitem{Art}{Kundenprojekt, Webseite}
\cvitem{Jahr}{2007-2012}
\cvitem{URL}{http://www.thelabelfinder.com}
\cvitem{Team}{Willi Schönborn \& Moritz Neuhäuser}
\cvitem{Technologien}{\textsc{Java}, \textsc{PHP}, \textsc{Solr}, \textsc{CodeIgniter}, \textsc{Hibernate}, \textsc{MySQL}}
\cvitem{Beschreibung}{TheLabelFinder ist eine online Mode-Suchmaschine, über die Benutzer ihre Lieblings-Modemarke in jeder Stadt der Welt finden können. Dabei wurde \textsc{Java} mit \textsc{Hibernate}, angeschlossen an eine \textsc{MySQL} Datenbank im Backend verwendet. Für das Frontend kam \textsc{Solr} als Suchindex und \textsc{NoSQL}-Datenspeicher zum Einsatz. Meine Hauptaufgaben in dem Projekt lag in der Programmierung des Backends und Bereitstellung der Daten in \textsc{Solr}.}

\vspace{10pt}

\newpage

%----------------------------------------------------------------------------------------
%	TECHNOLOGIES SECTION
%----------------------------------------------------------------------------------------

\section{Technologien}
Mit den folgenden Technologien arbeite ich täglich bis regelmäßig:
\newline

\cvitem{Sprachen}{\textsc{Java}, \textsc{SQL}, \textsc{JavaScript}, \textsc{PHP}, \textsc{HTML}, \textsc{Python}, \textsc{CSS}, \LaTeX, \textsc{Shell}}
\cvitem{Data Stores}{\textsc{Solr}, \textsc{MySQL}}
\cvitem{Libraries}{\textsc{jQuery}, \textsc{jQuery UI}}
\cvitem{Frameworks}{\textsc{Hibernate}, \textsc{JPA}, \textsc{Django}}
\cvitem{Testing}{\textsc{Junit}, \textsc{EasyMock}}
\cvitem{SCM}{\textsc{Git}}
\cvitem{Build Tools}{\textsc{Ant}, \textsc{Maven}}

\vspace{10pt}

Mit den folgenden Technologien habe ich in der Vergangenheit sehr intensiv gearbeitet:
\newline

\cvitem{Data Stores}{\textsc{PostgreSQL}}
\cvitem{Libraries}{\textsc{Guava}, \textsc{SLF4J}}
\cvitem{Frameworks}{\textsc{Guice}, \textsc{Servlet API}, \textsc{CodeIgniter}}
\cvitem{SCM}{\textsc{Subversion}}


%----------------------------------------------------------------------------------------
%	LANGUAGES SECTION
%----------------------------------------------------------------------------------------

\section{Sprachkenntnisse}

\cvitemwithcomment{Deutsch}{Muttersprache}{}
\cvitemwithcomment{English}{Fließend}{}

%----------------------------------------------------------------------------------------

\end{document}
